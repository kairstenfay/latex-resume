\documentclass[10pt]{article}

% graphicx package, useful for including eps, pdf, and raster graphics
\usepackage{graphicx}
\DeclareGraphicsExtensions{.png,.png,.jpg,.pdf}

% basic packages
\usepackage{fancyhdr}
\usepackage{color}
\usepackage{parskip}
\usepackage{float}
\usepackage{microtype}
\usepackage{url}
\usepackage{hyperref}
\usepackage{booktabs}
\usepackage{makecell}
\usepackage{multirow}
\usepackage{pbox}
\usepackage{enumitem}

% fonts
\usepackage{fontspec}
\setmainfont{DroidSans}[
  Extension   = .ttf,
  UprightFont = *,
  BoldFont    = *-Bold ]
\setmonofont{DroidSansMono}[
  Extension   = .ttf,
  UprightFont = * ]

\hypersetup{colorlinks,urlcolor=blue}
\urlstyle{same}

% page layout
\usepackage[top=1in, bottom=1in, left=0.75in, right=0.75in, includefoot, heightrounded]{geometry}

% Useful for seeing the layout as you tweak it.
%\usepackage{showframe}

\setlength{\headheight}{1.1in}
\pagestyle{fancy}
\fancyhf{}

\lhead{\includegraphics[height=1in,clip]{logo}}
\rhead{\pbox[b][1in][c]{\textwidth}{
    \small
    \textbf{Isinagawa ang pagsusuri ng sumusunod na laboratoryo:}\\
    Northwest Genomics Center\\
    University of Washington School of Medicine\\
    Genome Sciences Box 355065\\
    3720 15th Ave NE\\
    Seattle, WA 98195-5065}}

\newcommand{\PageLine}{\rule{\textwidth}{0.25mm}}

\newcommand{\link}[2]{\href{#1}{#2}\footnote{#1}}

\begin{document}

\begin{center}
\Large
\textbf{NWGC SARS-CoV-2 (COVID-19) Qual PCR Lab Report}
\end{center}

\bigskip

\begin{description}[font=\normalfont,align=left,labelwidth=12em]
\item [Pangalan ng Kalahok] \textbf{\VAR{pat_name|e}}
\item [Petsa ng Kapanganakan] \textbf{\VAR{birth_date|e}}
\item [Pagkakakilanlan ng ispesimen] \textbf{\VAR{qrcode|e}}
\item [Petsa ng Pagsusumite sa Sampol] \textbf{\VAR{collect_ts|e}}
\item [Petsa ng pagbibigay sa mga resulta]
  %- if status_code == "never-tested"
  \textbf{Hindi pa nasubukan}
  %- else
  \textbf{\VAR{result_ts|e}}
  %- endif
\item [Uri ng Ispesimen] Midnasal, kinolekta nang mag-isa
\end{description}

\PageLine

Mga Resulta ng SARS-CoV-2 (COVID-19) Qual PCR

%- if status_code == "negative"
\textbf{Hindi natukoy}\\
%- elif status_code == "positive"
\textbf{Natukoy ang SARS-CoV-2 (positibo)}\\
%- elif status_code == "inconclusive"
\textbf{Hindi kongklusibo}\\
%- elif status_code == "never-tested"
\textbf{Hindi nasuri}\\
%- else
\VAR{0/0} %# poor man's raise
%- endif

\PageLine

Pakitandaan:

\begin{itemize}
\item

  \textbf{Ang uri ng sampol at pagsusuri na ginamit sa proyektong ito ay
  isinasagawa bilang bahagi ng programa ng pagsubaybay sa kalusugan ng publiko
  sa pakikipagtulungan sa King County at iba pang ahensya.} Ang pagsusuring
  natanggap ninyo ay hindi kapalit sa pagtanggap ng personal na medikal na
  pangangalaga mula sa isang tagapagbigay ng pangangalagang pangkalusugan.

\item

  \textbf{Pagkakasunud-sunod ng Pag-iingat sa Ispesimen:} ang nasuring ispesimen
  ay kinolekta nang mag-isa, nilagyan ng label nang mag-isa, at isinumite sa
  SCAN (Seattle Coronavirus Assessment Network o Network ng Seattle sa Pagsusuri
  ng Koronabayrus) sa pamamagitan ng koreo o serbisyo sa paghahatid. Hindi ito
  kinolekta, nilagyan ng label, at ibiniyahe ng isang lisensyadong propesyonal
  sa pangangalagang pangkalusugan. Maaari itong makaapekto sa mga resulta ng
  pagsusuri, at maaaring hindi maibukod ang mga maling negatibong resulta.
  Maaaring kailanganin ng follow-up na klinikal na pagsusuri.

\item

  \textbf{Aparato sa pangongolekta ng ispesimen:} ang mid turbinate suwab ay
  hindi isang uri ng ispesimen na naaprubahan ng FDA (Food and Drug
  Administration o Administrasyon sa Pagkain at Gamot)/CDC (Centers for Disease
  Control and Prevention o Mga Sentro sa Pagkontrol at Pag-iwas sa Sakit) EUA
  (Emergency Use Authorization o Awtorisasyon sa Emerhensiyang Paggamit). Kaya
  maaaring hindi maibukod ng mga resultang hindi natukoy/negatibo ang impeksyong
  dulot ng SARS-CoV-2.

\item

  \textbf{Maaaring kumpirmahin ng Kagawaran ng Kalusugan ng Estado ng Washington
  ang mga ``positibong'' resulta.} Dapat talakayin ang resultang ito sa inyong
  tagapagbigay ng pangangalagang pangkalusugan kung mayroon kayong mga tanong.

\item{
  \textbf{Kung ``Hindi natukoy'' ang nakasaad sa inyong pagsusuri, ibig sabihin,
  hindi natukoy ang SARS-CoV-2 sa aming laboratoryo.}

  Walang perpektong pagsusuri at may maliit na posibilidad na negatibo ang
  maging resulta ng inyong pagsusuri kahit mayroon kayong SARS-CoV- 2 bayrus na
  nagdudulot ng COVID-19. Hindi ginagamit ng pagsusuring ito ang pamantayang
  paraan ng pangongolekta ng mga ispesimen ng isang tagapagbigay ng
  pangangalagang pangkalusugan, at ang sampol ay ipinadala sa pamamagitan ng
  koreo o serbisyo sa paghahatid, at hindi kinolekta at ibiniyahe sa isang lugar
  para sa pangangalagang pangkalusugan. Anuman ang maging resulta ng inyong
  pagsusuri, mainam na manatili kayo sa bahay kung masama ang inyong pakiramdam,
  at iwasan ninyong lumapit sa ibang tao. Kung kayo ay may iba pang problema sa
  kalusugan o may malalang sakit o nababahala tungkol sa inyong karamdaman,
  dapat kayong makipag-ugnayan sa inyong tagapagbigay ng pangangalagang
  pangkalusugan. Kung hindi mawawala ang inyong mga sintomas, dapat ninyong
  talakayin sa inyong tagapagbigay ng pangangalagang pangkalusugan ang tungkol
  sa pagkuha ng isa pang pagsusuri para sa COVID-19.
}

\item

  \textbf{Kung ``Hindi kongklusibo" ang inyong pagsusuri,} big sabihin, hind
  tiyak ang mga resulta ng aming pagsusuri sa laboratoryo para sa target na
  SARS-CoV-2.

  Walang perpektong pagsusuri at may maliit na posibilidad na hindi kongklusibo
  ang inyong resulta at mayroon kayong SARS-CoV-2 bayrus na nagdudulot ng
  COVID-19. Hindi ginagamit ng pagsusuring ito ang pamantayang paraan ng
  pangongolekta ng mga ispesimen ng isang tagapagbigay ng pangangalagang
  pangkalusugan, at ang sampol ay ipinadala sa pamamagitan ng koreo o serbisyo
  sa paghahatid, at hindi kinolekta at ibiniyahe sa isang lugar para sa
  pangangalagang pangkalusugan. Anuman ang maging resulta ng inyong pagsusuri,
  mainam na manatili kayo sa bahay kung masama ang inyong pakiramdam, at iwasan
  ninyong lumapit sa ibang tao. Kung kayo ay may iba pang problema sa kalusugan
  o may malalang sakit o nababahala tungkol sa inyong karamdaman, dapat kayong
  makipag-ugnayan sa inyong tagapagbigay ng pangangalagang pangkalusugan. Kung
  hindi mawawala ang inyong mga sintomas, dapat ninyong talakayin sa inyong
  tagapagbigay ng pangangalagang pangkalusugan ang tungkol sa pagkuha ng isa
  pang pagsusuri para sa COVID-19.

\item

  \textbf{Kung ``Hindi nasuri" ang nakasaad sa inyong pagsusuri,} ibig sabihin,
  walang sapat na ispesimen o nagkaroon ng pagkakamali sa proseso ng pagsusuri.

  Dahil hindi naisagawa ang pagsusuri, may posibilidad na mayroon kayong
  SARS-CoV-2 bayrus na nagdudulot ng COVID-19.  Hindi ginagamit ng pagsusuring
  ito ang pamantayang paraan ng pangongolekta ng mga ispesimen ng isang
  tagapagbigay ng pangangalagang pangkalusugan, at ang sampol ay ipinadala sa
  pamamagitan ng koreo o serbisyo sa paghahatid, at hindi kinolekta at ibiniyahe
  sa isang lugar para sa pangangalagang pangkalusugan.  Anuman ang maging
  resulta ng inyong pagsusuri, mainam na manatili kayo sa bahay kung masama ang
  inyong pakiramdam, at iwasan ninyong lumapit sa ibang tao. Kung kayo ay may
  iba pang problema sa kalusugan o may malalang sakit o nababahala tungkol sa
  inyong karamdaman, dapat kayong makipag-ugnayan sa inyong tagapagbigay ng
  pangangalagang pangkalusugan.  Kung hindi mawawala ang inyong mga sintomas,
  dapat ninyong talakayin sa inyong tagapagbigay ng pangangalagang pangkalusugan
  ang tungkol sa pagkuha ng isa pang pagsusuri para sa COVID-19.

\item

  \textbf{Iuulat ng pasilidad ng pagsusuri ang mga positibong pagsusuri (na
  ipinapakita bilang “NATUKOY” o “Natukoy ang Bayrus”) sa naaangkop na Kagawaran
  ng Kalusugan ng estado at county.} HINDI kailangang kayo mismo ang mag-ulat sa
  mga resultang ito.  Maaaring mag-follow up sa inyo ang Kalusugan ng Publiko
  kung positibo ang inyong pagsusuri para sa bagong koronabayrus upang tanungin
  kayo tungkol sa inyong karamdaman at sa mga taong maaaring nakaugnayan ninyo.

\item

  \textbf{Maaari ninyong i-printa o i-email ang mga resultang ito ng pagsusuri
  sa inyong tagapagbigay ng pangangalagang pangkalusugan.}

\end{itemize}

\bigskip
\textbf{Pamamaraan ng SARS-CoV-2 (COVID-19) Qual PCR:}

Idinisenyo ang pagsusuri na ito upang matukoy ang SARS-CoV-2 gamit ang
tiwaliing-transkripsyon PCR. Itong pagsusuri ay nakasalalay sa pagpapalakas ng
mga segmentong viral genome na Orf1B at S na may panloob pantaong RnaseP kontrol
ng pagpapalakas. Ang pagsusuri na ito ay binuo at ang mga katangian ng pagganap
nito ay tinukoy ng Northwest Genomics Center ng Universidad ng Washington. Ang
pagsusuri na binuo sa laboratoryo ay napatunayan at isinumite sa Administrasyon
ng Pagkain at Gamot (FDA) ng Estados Unidos para sa Awtorisasyon sa
Emerhensiyang Paggamit. Nakabinbin pa ang hiwalay na pagsusuri ng FDA para sa
pagpapatunay ng pagsusuri. Sertipikado ang laboratoryong ito sa ilalim ng Mga
Pagbabago sa Pagpapabuti sa Klinikal na Laboratoryo/Clinical Laboratory
Improvement Amendments (CLIA) na kwalipikadong magsagawa ng mga
napakakumplikadong klinikal na pagsusuri sa laboratoryo.

Para sa higit pang impormasyon tungkol sa tagapagbigay, pakitingnan ang mga link
na ito sa ibaba na ibinigay ng Kagawaran ng Kalusugan ng Estado ng Washington:

\begin{itemize}
\item

  \link{https://www.doh.wa.gov/Portals/1/Documents/1600/coronavirus/COVIDcasepositive.pdf}{Mga
  pasyenteng nakumpirma o pinaghihinalaang may COVID-19}

\item

  \link{https://www.doh.wa.gov/Portals/1/Documents/1600/coronavirus/COVIDexposed.pdf}{Mga
  pasyenteng nalantad sa isang kumpirmadong kaso ng COVID-19}

\item

  \link{https://www.doh.wa.gov/Portals/1/Documents/1600/coronavirus/COVIDconcerned.pdf}{Mga
  hindi nalantad na pasyenteng may mga sintomas ng COVID-19}

\end{itemize}

Makakakita ng mga karagdagang mapagkukunan sa Kagawaran ng Kalusugan ng Estado
ng Washington at sa mga website ng CDC:

\begin{itemize}
\item

  \url{https://www.doh.wa.gov}

\item

  \url{https://wwwn.cdc.gov/pubs/other-languages?Sort=Lang%3A%3Aasc&Language=Tagalog}

\end{itemize}

\end{document}
